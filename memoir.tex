\documentclass[12pt]{article} \usepackage[utf8x]{inputenc}
\usepackage{graphicx} \usepackage{multirow} \usepackage{hhline}
\usepackage{booktabs} \usepackage{vmargin} %cambia el margen
\usepackage{amsmath,amsthm} \usepackage{amsfonts} \usepackage{float}
\usepackage{listings}


\usepackage[hidelinks]{hyperref}




\title{
  Historia de las Matemáticas\\
  \large Historia de los algoritmos inspirados en la evolución:
  Algoritmos Genéticos, Programación Evolutiva, Estrategias de
  Evolución y Programación Genética.  }


\author{ 
  Francisco Luque Sánchez \\
  Ignacio Mas Mesa \\
  Miguel Morales Castillo \\
  María del Mar Ruiz Martín \\
}



\begin{document}
\maketitle
\begin{center}  
\includegraphics[scale=0.35]{escudo.png}
\end{center}

\newpage

\tableofcontents % para generar el índice de contenidos

\pagebreak

\section{Introducción}

Las metaheurísticas son una clase de estrategias de resolución de
problemas que aparecieron en los años 50. Este tipo de algoritmos se
inspiran en procesos naturales, físicos o sociales para intentar
resolver problemas, usualmente de optimización, cuando el espacio de
soluciones es demasiado amplio para ser estudiado por métodos
tradicionales.\\

En este trabajo abordaremos, desde una perspectiva histórica, el
origen, evolución y estado actual de este tipo de técnicas, así como
ejemplos prácticos de aplicación. En particular, incidiremos sobre una
clase importante de metaheurísticas, las cuales se inspiran en el
proceso natural de la evolución. Dentro de este grupo se engloban
cuatro enfoques principales, que son los algoritmos genéticos, la
programación evolutiva, la programación genética y las estatregias
de evolución.\\

\section{Metaheurísticas}

\section{Algoritmos genéticos}

\section{Programación evolutiva}

\section{Estrategias de evolución}

\section{Programación genética}

\section{Referencias}

\end{document}


